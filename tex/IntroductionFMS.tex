\documentclass[../main/FlatMarginalStability.tex]{subfiles} 
\begin{document}


\section{Introduction}

Marginal stability curves whose leading order growth near the critical wavenumber is higher than quadratic have been studied in the context of solidification processes\cite{proctor1991instabilities,riley1989eckhaus}.  Furthermore, these kinds of models may be relevent to understanding certain experimental data~\cite{}

In the present work, we take the following modified version of the Swift-Hohenberg equation~\cite{} as a model system to study the effects of a flattened marginal stability curve on the behavior of localized solutions.
\begin{equation}
u_t= r u-\left(1+\partial_{x}^2\right)^4 u+N[u]\label{eq:flatSH}.
\end{equation}
This equation describes the dynamics of a real field $u$ over one spatial dimension in time, where $N$ is some nonlinear function of $u$.  We have rescaled the equation so that the critical wavenumber that defines the natural wavelength of the patterned state is unity.  We will be interested in two possible choices of $N$ that allow for the existence of stable localized structures, namely $N_{23}[u]=bu^2-u^3$ and $N_{35}[u]=b u^3-u^5$.  The strength of the linear forcing term $r$ and the strength of the quadratic nonlinearity $b$ are left as parameters of the system.  

Localized states in the Swift-Hohenberg equation (SHE), which has the quadratic term $\left(1+\partial_{x}^2\right)^2u$ instead of the quartic term of Eq.~\ref{eq:flatSH}, have been studied in great detail  ~\cite{}. We consider the homotopy given in Eq.~\ref{eq:homSH} between the SHE and Eq.~\cite{eq:flatSH} in order to gain insight into how flattening the marginal stability curve changes the structure of the solutions.
\begin{equation}
u_t= r u-(1-\alpha)\left(1+\partial_{x}^2\right)^2 u-\alpha\left(1+\partial_{x}^2\right)^4 u+N[u]\label{eq:homSH},
\end{equation}
where $0\le \alpha \le 1$ so that we have the SHE when $\alpha=1$ and Eq.~\ref{eq:flatSH} when $\alpha=0$.  
Further insight can be gained by considering the formation of a quartic marginal stability curve from the collision of two quadratic minima as happens in the equation below when $\beta=0$.
\begin{equation}
u_t= r u-\left((1+\beta)^2+\partial_{x}^2\right)^2 \left((1-\beta)^2+\partial_{x}^2\right)^2 u+N[u]\label{eq:collSH},
\end{equation}
For both cases (Eq.~\ref{eq:homSH} , we perform linear and weakly nonlinear analysis in addition to presenting numerical continuation results computed with AUTO~\cite{}.     


There may be physical systems in which multiple patterned states with different characteristic wavelengths can exist simultaneously.  We would like to develop a model for such a system and study the interaction and competition between these patterned states.  We propose a modification to the Swift-Hohenberg equation in which such competition may be possible:
\begin{equation}
u_t= r u-\left(1+\partial_{x}^2\right)^2 \left[\left(q^2+\partial_{x}^2\right)^2+\delta \right] u+N[u]\label{eq:SHm}.
\end{equation}
Linear stability analysis of this equation results in a marginal stability curve (Fig. \ref{fig:marginalstability}) where two wavelengths can compete with each other.  The first wavelength is  $k=1$, just as in the original SHE (Eq.~\ref{eq:SH}), and the other occurs at 
\beqn
q_*=\frac{1}{2}\sqrt{1+3q^2 \pm\Delta}.
\eeqn
where $\Delta=\sqrt{(q^2-1)^2-8\delta}$ and $+$/ $-$ is used when $q>1$ and $q<1$ respectively.  We note that such solutions exist only when $(q^2-1)^2>8\delta$, which is the condition that there are  local extrema other than $k=1$ in the marginal stability curve.  We must further make the restriction that $\delta>-(q^2-1)^2$ in order that $k=1$ is in fact a local minimum.  If both these conditions are met, there will be a minimum at $q^*$ in addition to 1.  The curve that defines the region of interest to us is also proportional to the marginal stability curve of the original Swift-Hohenberg operator with characteristic wavenumber $q$ and forcing $\delta$.  \todo[inline]{Why is this the case, does it have some physical interpretation?}  This is not a constraining restriction since we will generally be interested in small values of $\delta$.
\FIGmarginalstability
When $\delta=0$, we have that $q_*=q$  and both wavelengths  become unstable at $r=0$. The instability of $q_*$ is shifted for nonzero values of $\delta$ by
\beqn
r_*=\frac{1}{128} \left(3( q^2-1)+\Delta\right)^2 \left( (q^2-1)^2- (q^2-1)\Delta+4 \delta \right).
\eeqn

We will be interested in the case that $\tfrac{8\delta}{(q^2-1)^2}<<1$,  so that the two competing wavelengths will have similar marginal stabilities and can thus compete.  This assumption allows us to approximate $q_*$ and $r_*$ as:
\beqa
q_*&\approx&q\left[1-\frac{\delta}{2q^2(q^2-1)^2}+\frac{\left(9 q^2-1\right) \delta ^2}{8 q^4 \left(q^2-1\right)^3}\right] \\
r_*&\approx& (q^2-1)^2 \delta-\delta^2
\eeqa
We note that the $q=1$ is a degenerate case that may need to be handled separately.  In terms of the our analysis here, $q=1$ requires that $\delta=0$ and in this case, $q_*=q$ and $r_*=0$.  Another interesting regime could be when $q$ is close to one so that we have nearby wavelengths competing. We might also want to consider the case when $q>>1$ (or $q<<1$) so that the wavelengths of the patterns occur on different lengthscales.  This regime is of interest for quasicrystals?

This equation has been studied by Bentley (2012), though with a slightly different parametrization.  The advantage of this parametrization over the one used by Bentley becomes apparent in the small $\delta$ limit.  We see that $q$ is approximately the wavenumber of the second competing pattern, and $\delta$ is proportional to the relative shift between the onsets of the two patterns.   We might also consider a slightly different parametrization in which the $\delta (1-k^2)^2$ is instead $\delta (1-k^2)$, the so-called "Proctor term" (I think).  The advantage of our parametrization is two-fold: (1) the characteristic wavenumber of one of the patterns is exactly one. (2) The relations between $q_*$, $r_*$ and $q$ and $\delta$ are much simpler in our case.    Bentley has looked at this equation as a model for magnetorotational Taylor-Couette flows.  His focus is in the supercritical regime where the patterned states bifurcate from the homogeneous state with a supercritical pitchfork bifurcation.  He is currently working to publish his work.  We don't know what additional work he has done beyond his thesis that his advisor has shared with us.


\end{document}
