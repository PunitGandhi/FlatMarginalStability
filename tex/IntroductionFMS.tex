\documentclass[../main/FlatMarginalStability.tex]{subfiles} 
\begin{document}


\section{Introduction}

Marginal stability curves whose leading order growth near the critical wavenumber is higher than quadratic have been studied in the context of solidification processes\cite{proctor1991instabilities,riley1989eckhaus}.  Furthermore, these kinds of models may be relevent to understanding certain experimental data~\cite{}

In the present work, we take the following modified version of the Swift-Hohenberg equation~\cite{} as a model system to study the effects of a flattened marginal stability curve on the behavior of localized solutions.
\begin{equation}
u_t= r u-\left(1+\partial_{x}^2\right)^4 u+N[u]\label{eq:flatSH}.
\end{equation}
This equation describes the dynamics of a real field $u$ over one spatial dimension in time, where $N$ is some nonlinear function of $u$.  We have rescaled the equation so that the critical wavenumber that defines the natural wavelength of the patterned state is unity.  We will be interested in two possible choices of $N$ that allow for the existence of stable localized structures, namely $N_{23}[u]=bu^2-u^3$ and $N_{35}[u]=b u^3-u^5$.  The strength of the linear forcing term $r$ and the strength of the quadratic nonlinearity $b$ are left as parameters of the system.  

Localized states in the Swift-Hohenberg equation (SHE), which has the quadratic term $\left(1+\partial_{x}^2\right)^2u$ instead of the quartic term of Eq.~\ref{eq:flatSH}, have been studied in great detail  ~\cite{}.   We consider the homotopy given in Eq.~\ref{eq:homSH} between the SHE and Eq.~\cite{eq:flatSH} in order to gain insight into how flattening the marginal stability curve changes the structure of the solutions.
\begin{equation}
u_t= r u-(1-\alpha)\left(1+\partial_{x}^2\right)^2 u-\alpha\left(1+\partial_{x}^2\right)^4 u+N[u]\label{eq:homSH},
\end{equation}
where $0\le \alpha \le 1$ so that we have the SHE when $\alpha=1$ and Eq.~\ref{eq:flatSH} when $\alpha=0$.  
Further insight can be gained by considering the formation of a quartic marginal stability curve from the collision of two quadratic minima as happens in the equation below when $\beta=0$.
\begin{equation}
u_t= r u-\left((1+\beta)^2+\partial_{x}^2\right)^2 \left((1-\beta)^2+\partial_{x}^2\right)^2 u+N[u]\label{eq:collSH},
\end{equation}
For both cases (Eq.~\ref{eq:homSH} , we perform linear and weakly nonlinear analysis in addition to presenting numerical continuation results computed with AUTO~\cite{}.     


\end{document}
