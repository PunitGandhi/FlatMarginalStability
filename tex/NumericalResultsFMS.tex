\documentclass[../main/FlatMarginalStability.tex]{subfiles} 




\begin{document}





\section{Numerical results}
We set the nonlinearity parameter to $b=1.8$ as was done by Burke et. al for the original SHE, and perform numerical continuation of the time steady solutions of Eq.~\ref{eq:flatSH}.   We notice some interesting differences in the structure of the bifurcation diagram as well as in the localized solutions themselves.  We first give some general results on the structure and types of solutions found for this equation with a focus on interpreting the results as multi-pulse states.  We then present some numerical results of the homotopy connecting the modified SHE to the original SHE (Eq.~\ref{eq:homSH}).  Finally, we show some numerical results for Eq.~\ref{eq:collSH}.  \todo{do some numerical continuation of this}.


\subsection{Bifurcation diagram and structure of solutions for the modified SHE}
The basic bifurcation diagram structure of Eq.~\ref{eq:flatSH} in the vicinity of $r=0$ (shown in Fig.~\ref{fig:bifurctaiondiagram1}) is markedly different from that of the original SHE.
\FIGbifurcationdiagramA
The periodic branch bifurcating from the homogeneous solution at $r=0$ has 20 periods and will be called $P20$.  The first secondary bifurcation from $P20$, forms a snaking branch of one-pulse localized states like the ones shown in Fig.~\ref{fig:snaking1}.
\FIGsnakingA
This branch eventually reconnects to another periodic branch that does not have a constant amplitude and has 12 periods in the domain (See Fig.~\ref{fig:doubleperiod1}).
\FIGdoubleperiod
This periodic solution does not bifurcate from the homogeneous solution, but from the $P24$ periodic branch.  It also has a wiggle in it, where the solution structure inverts so that the small oscillation are on the top instead of the bottom. The bifurcation is  actually the third branch point along the $P24$ branch if followed starting from the homogeneous solution.  The previous two branches along $P24$ are quite messy with seemingly randomly positioned pulses (See Fig.~\ref{fig:snakingmess1} for an example).
\FIGsnakingmess  
Additionally, we have looked at the solutions formed at the next two bifurcation points along the $P20$ periodic branch.  The second branch snakes with two-pulse localized states while the third branch is messy with seemingly randomly distributed pulses.

Another example of where a  multipulse state branch connects to a secondary periodic state branch is shown in Fig.~\ref{fig:bifurcationdiagram2}.  In this case, a five pulse state grows into the secondary periodic state upon each pulse doubling in size.  Note that this situation is slightly different in that both branches come from the same primary periodic branch.  It is also interesting to note that another five pulse branch appears below the periodic ( or ten pulse) branch, but does not connect as far as I've followed it. The secondary periodic branch with 10 periods shown in this figure contains a loop near $r=0.3$ just as the secondary periodic branch with 12 periods did in Fig.~\ref{fig:bifurcationdiagram1}.  
\FIGbifurcationdiagramB

There are several periodic solutions branching from the homogenous state between $P20$ and $P24$.  By looking at the secondary branches from these periodic states, we hope to see a pattern start to emerge with the location of multipulse states.  Note that we only track multipulse states in which all pulses are identical.  The table below lists in order, the periodic solutions along the diagram from in order of increasing forcing.  We would like to note that we can also interpret the periodic solution shown in Fig.~\ref{fig:doubleperiod1} as a twelve pulse state that just happens to exactly fill the simulation domain.  There are several solutions of this type that can be seen as a multipulse state with half the number of pulses as the periodic branch from which it emerges, or as a periodic solution with twice the wavelength of the branch from which it emerges.  



\begin{table}[ht]
\caption{Cascade of secondary multipulse branches that bifurcate from periodic states.} % title of Table
\centering % used for centering table
\begin{tabular}{c | c c c c c c c c c c} % centered columns (4 columns)
\hline\hline %inserts double horizontal lines
 Branch \# &\multicolumn{10}{c}{ Primary branch}\\
\hline  %\\  %[0.5ex] % inserts table 
%heading
10 	&\textbf{10p} &??		&??		&??		&??		&??		&??		&??		&??		&??	\\
9	&      		&??		&??		&??	 	&??		&??		&??		&??		&??		&??	\\
8	&4p  		&		&		&??	 	&??		&??		&??		&??		&??		&??	\\
7	&      		&		&3p		&??	 	&\textbf{11p}	&??		&??		&??		&??		&??	\\
6	& 2p    		&		&		&\textbf{9p} 	&2p		&??		&??		&		&		&??	\\
5	&\textbf{5p}  	&		&\textbf{7p}	&2p	 	&		&		&		&4p		&		&??	\\
4	&\textbf{4p} 	&		&3p		&	 	&2p		&		&		&		&3p		&??	\\
3	&      		&		&		&\textbf{6p} 	&		&		&		&2p		&		&\textbf{12p}	\\
2	&\textbf{2p} 	&		&		&	 	&2p		&		&		&		&5p		&	\\
1	&\textbf{1p}  	&		&\textbf{3p}	&2p	 	&		&		&		&\textbf{8p}	&3p		&2p	\\
\hline % inserts single horizontal line
 	&$P20$	&$P19$	&$P21$	&$P18$	&$P22$	&$P17$	&$P23$	&$P16$	&$P15$	&$P24$ \\ [1ex] % [1ex] adds vertical spac
\hline %inserts single line
\end{tabular}
\label{table:multipulse} % is used to refer this table in the text
\end{table} 

Following the homogeneous solution branch, we see a series of periodic states bifurcate as the forcing increases.  Each of these branches is labeled by the number of wavelengths that fit into the domain, and the sequence is shown in Table \ref{table:multipulse}.  Secondary bifurcations occur along each of these periodic branches as they are followed away from the homogeneous solution.  Studying the solutions along these secondary branches shows a cascade of multipulse solutions that have a very regular pattern.  The first periodic branch, $P20$ has  one-pulse solution branching off of the first bifurcation point, and then a two pulse solution off of the second one.  The pattern continues as you continue up the branch, except that the nth bifurcation produces an n-pulse state only if n is a factor of the number of periods of  on the branch (namely 20).  The next periodic branch begins with a two-pulse state at the first bifurcation point and increases as one follows the periodic branch up.  The pattern continues so that the next branch starts with a three-pulse state and increases.  Again the multipulse states only exist as multipulse states if the number of pulses that should be there is a factor of the period.  There are other, more complicated multipulse states that appear when the number of pulses that should appear a a particular bifurcation point has greatest common factor with the number of periods on the homogeneous branch.  In this case, the number of pulses on the branch will be the greatest common factor and this can be understood as effectively simulating only a fraction of the domain.  For example, an 8 pulse state should appear at the 8th bifurcation along $P20$, but this does not happen since 8 is not a factor of 20.  Instead, a 4-pulse state appears, which could be considered an 8 pulse state on a 40 period domain.  So, in effect, we've only simulated half of the domain resulting in only half of the expected number of pulses.  These multipulse states are indicated on the table in not-bold font, and generally seem to be more complicated patterns than the standard multipulse states.  \todo[inline]{ Note that the question marks on the table indicate that these branches have not yet been computed.  This still needs to be done.  The other thing I should do is to mark on the table where the separation between the bifurcation points near the primary bifurcation and the bifurcation points near the saddle-node bifurcation. }  

This pattern of increasing number of pulses seems to hold for the secondary branch points on the lower part of the periodic branch, and then repeats itself starting near the saddle-node bifurcation of the branch. All of the secondary branches were followed from the $P20$ periodic branch up to $r=0.4$, and the multipulse states are summarized in table~\ref{table:P20}. \todo[inline]{It may be possible that the algorithm missed some branch points if they were too close together, so I should try to rerun the calculation with smaller stepsizes just to be sure.   I should also follow these branches further along to see where they connect to.}. The pattern for the lower branch exactly matches the pattern for the upper branch, where the first branch point corresponds to the twelfth. The corresponding branches do not, however, seem to connect (as far as I have followed them).  The fifth branch and 16th branch, for example, both have 5 pulse states, but the two branches do not seem to connect.  The fourth and 8th branches are 4 pulse states that connect.  Similarly, the 15th and 19th branches connect with each other, but do not seem to connect to the earlier 4 pulse branch. The one pulse state at the 12th branch point has is a pulse within a patterned state, and doesn't really grow by adding periods in the same way that the first one pulse solution does.  It might be a little bit of a stretch to call it a one pulse state.

\begin{table}[ht]
\caption{Secondary multipulse branches that bifurcate from the $P20$ periodic state.} % title of Table
\centering % used for centering table
\begin{tabular}{c | c } % centered columns (4 columns)
\hline\hline %inserts double horizontal lines
 Branch \# &  Multipulse state \\
\hline  %\\  %[0.5ex] % inserts table 
%heading

\hline % inserts single horizontal line
1 & 1p \\
2& 2p\\
4 & 4p\\
5 & 5p\\
6 & 2p\\
8 & 4p\\
10 &10p\\
\hline%inserts single line
12 & 1p\\
13 & 2p\\
15 & 4p\\
16 & 5p\\
17 & 2p\\
19 & 4p\\
21 & 10p\\
[1ex] % [1ex] adds vertical spac
\hline %inserts single line
\end{tabular}
\label{table:P20} % is used to refer this table in the text
\end{table} 

Another surprising result is seen in the first secondary branch from $P19$ - a snaking pattern  begins  after a messy start along the branch.  The snaking pattern involves 3 pulses of unequal size, and grows by first loosing fronts on the central pattern and then gaining fronts on all three patterns. \todo[inline]{I don't think front is the right word, but I hope it is clear what is meant.} The pattern eventually fills the domain, except that defects remains.  The branch becomes messy again at this point as the defects shift around and change slightly.  The snaking bifurcation diagram is shown in Fig.~\ref{fig:bifurcationdiagram19a}, and some sample solutions along the snaking region are shown in Fig.~\ref{fig:snaking19a}.
\FIGbifurcationdiagramC
\FIGsnakingB

\subsection{Connecting the modified SHE to the original SHE}

For the original SHE, the localized solutions that grow from an amplitude modulation of the periodic solution have a single repeated pattern.  The localized solutions of the modified SHE, on the other hand, have a structure of alternating large and small patterns was shown in Fig.~\ref{fig:doubleperiod1}.  We use the homotopy defined by Eq.~\ref{eq:homSH} to help us numerically gain insight into how this new kind of localed solution forms.  We show the  P20 and the snaking branch emerging from the first secondary bifurcation for a series of values of $\alpha$ in Fig.~\ref{fig:homsh1}.
\FIGhomshA
  A transition from the alternating localized pattern to the single repeating localized pattern occurs somewhere between $\alpha=.1$ and $\alpha=.3$.  In Fig.~\ref{fig:homsh2}, we show the same bifurcation diagram for $\alpha=.15,.2,.25$ in order to capture more details of the transition.   
\FIGhomshB
It is possible that this transition occurs because the spatial eigenvalues of the linear problem initially have nearby imaginary parts that can cause beating in addition to nonzero real parts wich cause modulation.   A series of solutions along this branch near $r=0$ is shown in Fig.~\ref{fig:modbeat} to indicate the possibility that this is the mechanism that causes alternating size of the patterns on the localized branch.  
\FIGmodbeat


\end{document}