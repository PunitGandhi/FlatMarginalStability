\documentclass[../main/FlatMarginalStability.tex]{subfiles} 
\begin{document}



\section{Weakly Nonlinear Analysis}
We would like to look at small amplitude solutions in the neighborhood of the $r=0$ bifurcation where the periodic state branches off of the homogeneous state in the space of steady-state solutions.  We will take a multi-scale approach, and start by writing the the forcing strength as $r=\epsilon^2 \mu$, where $\epsilon$ is an arbitrary small parameter.  With this choice of forcing, an analysis of the scaling of the eigenvalues of the linear problem indicate that we need a slow timescale $T=\epsilon^2 t$, and long spatial scale $X=\epsilon^{1/2} x$.   The derivatives then become $\partial_t \rightarrow \partial_t+\epsilon^2\partial_T$ and $\partial_x \rightarrow \partial_x+\epsilon\partial_X$.  We will assume that the system will not change on the fast timescale, so we can neglect the $\partial_t$ term. 

The linear part of the modified Swift-Hohenberg equation (Eq.~\ref{eq:flatSH}), 
\beqn
L= r-\left(1+\partial_{x}^2\right)^4,
\eeqn
can be expanded as $L=L_0+\epsilon^{1/2} L_{1/2}+\epsilon L_1+\epsilon^{3/2} L_{3/2}+\epsilon^2 L_2+...$ where:
\begin{subequations}
\begin{align}
L_0 =& -\left(1+\partial_x^2\right)^4  \\
L_{1/2} =& -8\left(1+\partial_x^2\right)^3 \partial_x\partial_X \\
L_1 =&- 4 \left[7 \partial_x^6+15\partial_x^4+9 \partial_x^2+\right] \partial_X^2\\  
L_{3/2} =& -8   \left[ 7 \partial_x^4+10  \partial_x^2+9\right]\partial_x \partial_X^3 \\
L_2 =& -2\left[ 35 \partial_x^4+30 \partial_x^2 +3\right]\partial_X^4+\mu-\partial_T  %\\
%L_{5/2} =& -4 \left[ \left(3(q^2+1)+14 \partial_x^2\right)\partial_X^4+\delta(1 +\partial_x^2) \right] \partial_x \partial_X  \\
%L_{3} =& -2 \left[ \left(q^2+1+14 \partial_x^2\right)\partial_X^4 +\delta(1 +3 \partial_x^2) \right] \partial_X^2 
\end{align}
\end{subequations}

\subsection{The quadratic-cubic nonlinearity}
We will first consider the case when $N=N_{23}$ so that the modified Swift-Hohenberg equation can be written as $L[u]+N_{23}[u]=0$.  We will assume that the solution can be written as an asymptotic series with the leading term of order $\epsilon$, namely $u=\epsilon u_1 + \epsilon^{3/2} u_{3/2} +\epsilon^2 u_2+...$ 

We can then write out the resulting equation at each order of $\epsilon$ by matching terms at the proper order.
\begin{subequations}
\begin{align}
\mathcal{O}(\epsilon): \:  &-L_0 u_1 =0
\label{eq:msh23o1b} \\
\mathcal{O}(\epsilon^{3/2}): \: &-L_0 u_{3/2} = L_{1/2} u_1 
\label{eq:msh23o15b} \\
\mathcal{O}(\epsilon^2): \:  &-L_0 u_2 = L_{1/2} u_{3/2} +L_1 u_1 +b u_1^2
\label{eq:msh23o2b}\\
\mathcal{O}(\epsilon^{5/2}): \:  &-L_0 u_{5/2} = L_{1/2} u_{2} +L_1 u_{3/2}+ L_{3/2} u_1 +2b u_1 u_{3/2}
\label{eq:msh23o25b}\\
\mathcal{O}(\epsilon^{3}): \:  &-L_0 u_{3} = L_{1/2} u_{5/2} +L_1 u_{2}+ L_{3/2} u_{3/2}+L_2 u_1   +b u_{3/2}^2+2b u_1 u_2 -u_1^3
\label{eq:msh23o25b}
\end{align}
\end{subequations}

The solution to the $\mathcal{O}(\epsilon)$ equation can be expressed in terms of the yet to be determined complex amplitude $A_{11}$ as:
\beqn
u_1(x,X,T)=A_{11}(X,T)e^{i x} +c.c.
\label{eq:sol23o1}
\eeqn
Furthermore, since $L_{1/2} u_1=0$, we can absorb $u_{3/2}$ into $u_1$ as a correction. 

For the next order in $\epsilon$,$\mathcal(\epsilon^2)$ , we see that $L_{1/2} u_{3/2}$ vanishes, and will assume the following form for $u_2$:
\beqn
u_2(x,X,T)=C_{20}(X,T)  + \left[ A_{21}(X,T)e^{i x}+A_{22}(X,T)e^{2 i x}  +c.c.\right]\nonumber
\label{eq:sol23o2}
\eeqn

The resulting condition becomes:
\beqn
0=2 b |A_{11}|^2- C_{20} +\left(b A_{11}^2-3^4 A_{22}\right)e^{2 i x}  
\label{eq:solvability2}
\eeqn
which we can use to determine the constants $C_{20}$ and $A_{22}$ in terms of $A_{11}$.  


\end{document}